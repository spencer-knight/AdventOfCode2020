\documentclass{article}
\usepackage[utf8]{inputenc}
\usepackage{amsmath}

\title{Advent of Code 2020 Day 10 Part 2 Calculation Time}
\author{Spencerk226}
\date{December 2020}

\begin{document}

\maketitle

\section{Solving With Pure Recursion}
\quad The maximum number of calculations for all combinations of a set of adapters n adapters long when using pure recursion can be represented by 
\begin{equation*}
    f(n) = \sum_{i = 0}^n 3^i
\end{equation*}
\quad One computation is not necessarily one clock cycle but to make it easy I will assume that it is.
With my input the set was 104 adapters long, so the number of calculations equates to approximately $6.26*10^{49}$ calculations. 
\newline
\newline
\quad With my processor running a 4GHz we can calculate the time in years it will take to finish processing as:
\begin{equation*}
    \frac{1\,ns}{4\,computations} * \frac{6.26*10^{49}\,computations}{1} * \frac{1\, s}{1*10^9\,ns} * \frac{1\,yr}{3.15*10^7\,s}
\end{equation*}
which equates to $4.98*10^{32}$ years, this is put into perspective when you note that the universe is approximated to be $1.38*10^{10}$ years old.
\newline
\newline
\quad I think it is safe to say that if you are doing this with straight up recursion you won't finish today's challenge before the 25th.
\section{Solving with Recursion Using Memoization}
When using memoization the number of calculations can be represented by approximately
\begin{equation*}
    f(n) = \sum_{i = 0}^n n + 1
\end{equation*}
Which would be 105 calculations for me, which can be done in a few nanoseconds, in reality it took about 3000 microseconds.
\end{document}
